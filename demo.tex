\documentclass[10pt]{beamer}
\usepackage{amsmath}
\usepackage{mathdots}
\usetheme[progressbar=frametitle]{metropolis}
\usepackage{booktabs}
\usepackage[scale=2]{ccicons}
\usepackage{pgfplots}
\usepackage{amsmath}
\newcommand{\norm}[1]{\left\lVert#1\right\rVert}
\usepgfplotslibrary{dateplot}
\hypersetup{
    colorlinks=true,
    linkcolor=blue,
    filecolor=magenta,      
    urlcolor=cyan,
}
 
\urlstyle{same}

\usepackage{xspace}
\newcommand{\themename}{\textbf{\textsc{metropolis}}\xspace}

\title{A Fault Detection method based on SVM technique with $T^2$-Statistic and Riemannian metric}
%\subtitle{going beyond IF and Scopus index (v2.0)}
\author{Zhongcheng Dai  \\
 Supervisor: M.Sc. Han Yu 
}
 \date{30 November 2019}
\institute{Automatic Control and Complex Systems}
\titlegraphic{\hfill\includegraphics[height=1cm]{logo_ITB}}
\begin{document}
\maketitle
\begin{frame}{contents}
  \setbeamertemplate{section in toc}[sections numbered]
  \tableofcontents[hideallsubsections]
\end{frame}
\section{Introduction}
\begin{frame}{Background}
\metroset{block=fill}
    \begin{exampleblock}{Traditional methods}
	\begin{itemize}
    \item $T^2$-statistic
    \item Q-statistic
    \item Kullback-Leibler divergence
    \item The Wishart distribution-based methods
    \item \dots \dots
    \end{itemize}
    \end{exampleblock}
\end{frame}
\begin{frame}{Background}
 A brief summary of the steps:
      \begin{itemize}
      \item \textcolor{blue}{S1}: collect sufficient fault-free data
      \item \textcolor{blue}{S2}: offline train to determine the threshold
      \item \textcolor{blue}{S3}: use the threshold to check the new data
 	 \end{itemize}  
\end{frame}
\begin{frame}{Background}
 \begin{exampleblock}{merits of the traditional methods}
	\begin{itemize}
    \item The detection performance for additive faults is practical
    \item Online calculations are small, only need to compare threshold
    \end{itemize}
    \end{exampleblock}
\end{frame}
\begin{frame}{Background}
\begin{exampleblock}{Drawbacks of traditional methods}
      \begin{itemize}
      \item The detection performance for multiplicative faults is unqualified
      \item Unrecognized case where the variance becomes smaller
 	 \end{itemize}  
 	 \end{exampleblock}
\end{frame}

\begin{frame}{Motivation and solution}
\begin{exampleblock}{Motivation}
      \begin{itemize}
      \item Eliminate the shortcomings of traditional methods
      \item Enhance the fault detection performance
 	 \end{itemize}  
 	 \end{exampleblock}
 \begin{exampleblock}{Solution}
      \begin{itemize}
      \item $T^2$-statistic and Riemannian metric
      \item Support Vector Machine
 	 \end{itemize}  
 	 \end{exampleblock}

\end{frame}






\section{Basic knowledge}

\begin{frame}{Data}
The data set $\textbf{y}$ is given, which has $n_1$ normal data and $n_2$
faulty data
\begin{equation}
   y = [y_1 \ y_2 \ \dots y_{n_1} \ y_{n_1+1} \ y_{n_1+2} \ \dots y_{n_1+n_2}] 
\end{equation}
Where $n_1 + n_2 = N$
\begin{equation}
    y(i) = 
    \begin{pmatrix}
        y_1(i) \\
        y_2(i) \\
       \vdots  \\
        y_m(i) \\ 
    \end{pmatrix}
\end{equation}
Where $\textbf{m}$ is the dimension of the output data. % it is also called "degrees of freedom" in $T^2$ distribution.
\end{frame}


\begin{frame}{$T^2$-statistic}
    \begin{columns}
        \begin{column}{0.5\textwidth}
            the steps of $T^2$-statistic:
      \begin{itemize}
      \item \textcolor{blue}{S1}: computing the mean
      \item \textcolor{blue}{S2}: computing the variance
      \item \textcolor{blue}{S3}: Normalizing the data
      \item \textcolor{blue}{S4}: computing the  covariance
      \item \textcolor{blue}{S5}: computing the value of evaluation function
 	 \end{itemize}  
        \end{column}
        \begin{column}{0.5\textwidth}  %%<--- here
               \begin{equation}
               E(y) \approx \frac{1}{n_1}\sum_{i=1}^{n_1}y(i) 
               \end{equation}
                \begin{equation}
                   \sigma^2 \approx \frac{1}{n_1-1}\sum_{i=1}^{n_1}y^2(i) 
                \end{equation}
                \begin{equation}
                Y = \sigma^{-1}(y-E(y))
                \end{equation}
                \begin{equation}
                \Sigma_y \approx \frac{YY^T}{N-1}
                \end{equation}
                \begin{equation}
                       J_{T^2}(i) = y_i^T\Sigma_y^{-1}y_i
                \end{equation}
        \end{column}
    \end{columns}
\end{frame}

\begin{frame}{$T^2$-statistic}
Where i = 1,2 \dots N. 
  $
                \sigma^{-1} \approx
                \begin{bmatrix}
                \sigma_1^{-1} &&&0\\
                &  \sigma_2^{-1} && \\
                && \ddots & \\
                0&&& \sigma_m{-1}
                \end{bmatrix}
  $ 
  \par
  In machine learning, this process can be called feature extraction.
  Feature:$ J_{T^2}(i)$
\end{frame}

\begin{frame}{Riemannian metric}
    \metroset{block=fill}
    \begin{exampleblock}{Common Riemannian manifolds}
	\begin{itemize}
    \item \textcolor{red}{symmetric positive definite (SPD) matrix manifolds}
    \item Grassmann manifolds
    \item Stiefel manifolds
    \end{itemize}
    \end{exampleblock}
\end{frame}


\begin{frame}{Geometry of SPD Matrices}
An n $\times$ n square matrix M is SPD matrix, if and only if
\begin{itemize}
    \item $M^T = M$
    \item $u^TMu>0$ and $\forall u \neq 0$
\end{itemize}
SPD matrices have the following properties:
\begin{itemize}
\item 1. $\forall$ M $\in$ SPD(n), $M^{-1}$ $\in$ SPD(n) i.e., SPD matrices are invertible. 
\item 2. $\forall$ M $\in$ SPD(n), eigenvalues are positive i.e., λ(M) > 0.
\end{itemize}
\end{frame}

\begin{frame}
    \frametitle{Riemannian metric}
    \begin{columns}
        \begin{column}{0.3\textwidth}
           some text here some text here some text here some text here some text here
        \end{column}
        \begin{column}{0.7\textwidth}  %%<--- here
            \begin{figure}[!htb] 
                \centering 
                \includegraphics[width=0.7\textwidth]{RieDis.png}
                \caption{\href{https://www.mdpi.com/1424-8220/19/2/379}{Riemannian mainfold}} 
            \end{figure}
        \end{column}
    \end{columns}
\end{frame}

\begin{frame}{Riemannian distance}
    \begin{equation}
R_d(C_1,C_2) = \norm{ C_1^{-\frac{1}{2}}C_2C_1^{-\frac{1}{2}}}
    \end{equation} 
\end{frame}




\begin{frame}{Support Vector Machine}

\end{frame}


\section{The Algorithm to find the the threshold}


\begin{frame}{Riemannian metric}
\end{frame}


\section{Benchmark study}

\begin{frame}{algorithm}
    
\end{frame}



\section{Conclusions and future work}

\begin{frame}{take home notes}

\begin{center}science is about:\end{center}

\begin{center}
\begin{enumerate}
	\item \textcolor{blue}{honesty} in researching the problem
	\item \textcolor{blue}{bravery} in publishing the results 
    \item \textcolor{blue}{big heart} in getting feedback 
\end{enumerate}
\end{center}
\end{frame}

\begin{frame}[standout]
\begin{center}
Thank you for your attention!
\end{center}
\end{frame}

\end{document}
